%% Generated by Sphinx.
\def\sphinxdocclass{report}
\documentclass[letterpaper,10pt,english]{sphinxmanual}
\ifdefined\pdfpxdimen
   \let\sphinxpxdimen\pdfpxdimen\else\newdimen\sphinxpxdimen
\fi \sphinxpxdimen=.75bp\relax
\ifdefined\pdfimageresolution
    \pdfimageresolution= \numexpr \dimexpr1in\relax/\sphinxpxdimen\relax
\fi
%% let collapsible pdf bookmarks panel have high depth per default
\PassOptionsToPackage{bookmarksdepth=5}{hyperref}

\PassOptionsToPackage{booktabs}{sphinx}
\PassOptionsToPackage{colorrows}{sphinx}

\PassOptionsToPackage{warn}{textcomp}
\usepackage[utf8]{inputenc}
\ifdefined\DeclareUnicodeCharacter
% support both utf8 and utf8x syntaxes
  \ifdefined\DeclareUnicodeCharacterAsOptional
    \def\sphinxDUC#1{\DeclareUnicodeCharacter{"#1}}
  \else
    \let\sphinxDUC\DeclareUnicodeCharacter
  \fi
  \sphinxDUC{00A0}{\nobreakspace}
  \sphinxDUC{2500}{\sphinxunichar{2500}}
  \sphinxDUC{2502}{\sphinxunichar{2502}}
  \sphinxDUC{2514}{\sphinxunichar{2514}}
  \sphinxDUC{251C}{\sphinxunichar{251C}}
  \sphinxDUC{2572}{\textbackslash}
\fi
\usepackage{cmap}
\usepackage[T1]{fontenc}
\usepackage{amsmath,amssymb,amstext}
\usepackage{babel}



\usepackage{tgtermes}
\usepackage{tgheros}
\renewcommand{\ttdefault}{txtt}



\usepackage[Bjarne]{fncychap}
\usepackage{sphinx}

\fvset{fontsize=auto}
\usepackage{geometry}


% Include hyperref last.
\usepackage{hyperref}
% Fix anchor placement for figures with captions.
\usepackage{hypcap}% it must be loaded after hyperref.
% Set up styles of URL: it should be placed after hyperref.
\urlstyle{same}

\addto\captionsenglish{\renewcommand{\contentsname}{Contents:}}

\usepackage{sphinxmessages}
\setcounter{tocdepth}{1}



\title{AIM\sphinxhyphen{}CU}
\date{Jan 30, 2025}
\release{1.0.0}
\author{Smriti Prathapan, Berkman Sahiner, Dhaval Kadia, Ravi K.\@{} Samala}
\newcommand{\sphinxlogo}{\vbox{}}
\renewcommand{\releasename}{Release}
\makeindex
\begin{document}

\ifdefined\shorthandoff
  \ifnum\catcode`\=\string=\active\shorthandoff{=}\fi
  \ifnum\catcode`\"=\active\shorthandoff{"}\fi
\fi

\pagestyle{empty}
\sphinxmaketitle
\pagestyle{plain}
\sphinxtableofcontents
\pagestyle{normal}
\phantomsection\label{\detokenize{index::doc}}


\sphinxAtStartPar
A CUSUM\sphinxhyphen{}based tool for AI Monitoring

\sphinxAtStartPar
AIM\sphinxhyphen{}CU is a statistical tool for AI monitoring using cumulative sum (AIM\sphinxhyphen{}CU). AIM\sphinxhyphen{}CU computes:
\begin{itemize}
\item {} 
\sphinxAtStartPar
The parameter choices for change\sphinxhyphen{}point detection based on an acceptable false alarm rate

\item {} 
\sphinxAtStartPar
Detection delay estimates for a given displacement of the performance metric from the target for those parameter choices.

\end{itemize}


\chapter{Code execution}
\label{\detokenize{index:code-execution}}
\sphinxAtStartPar
Clone AIM\sphinxhyphen{}CU repository.

\begin{sphinxVerbatim}[commandchars=\\\{\}]
git\PYG{+w}{ }clone\PYG{+w}{ }https://github.com/DIDSR/AIM\PYGZhy{}CU.git
\end{sphinxVerbatim}

\sphinxAtStartPar
Run the following commands to install required dependencies (Python = 3.10 is used).

\begin{sphinxVerbatim}[commandchars=\\\{\}]
apt\PYGZhy{}get\PYG{+w}{ }\PYGZhy{}y\PYG{+w}{ }install\PYG{+w}{ }python3
apt\PYGZhy{}get\PYG{+w}{ }\PYGZhy{}y\PYG{+w}{ }install\PYG{+w}{ }pip
\PYG{n+nb}{cd}\PYG{+w}{ }AIM\PYGZhy{}CU
pip\PYG{+w}{ }install\PYG{+w}{ }\PYGZhy{}r\PYG{+w}{ }requirements.txt
\end{sphinxVerbatim}

\sphinxAtStartPar
Run AIM\sphinxhyphen{}CU.

\begin{sphinxVerbatim}[commandchars=\\\{\}]
python3\PYG{+w}{ }app.py
\end{sphinxVerbatim}

\sphinxAtStartPar
Open the URL \sphinxurl{http://0.0.0.0:7860} that is running the AIM\sphinxhyphen{}CU locally.


\chapter{Demo}
\label{\detokenize{index:demo}}
\sphinxAtStartPar
AIM\sphinxhyphen{}CU can also be run through the demo available at \sphinxurl{https://huggingface.co/spaces/didsr/AIM-CU}. If Space is paused, click on Restart button.

\sphinxstepscope


\section{Methods}
\label{\detokenize{ref_method:methods}}\label{\detokenize{ref_method::doc}}
\sphinxAtStartPar
A two\sphinxhyphen{}sided CUSUM control chart computes the cumulative differences or
deviations of individual observations from the target mean (or
in\sphinxhyphen{}control mean, \(\mu_{in}\)). The positive and negative cumulative
sums are calculated:
\begin{equation*}
\begin{split}\\ S_{hi}(d) = max(0, S_{hi}(d-1) + x_d  - \hat{\mu}_{in} - K)
\\ S_{lo}(d) = max(0, S_{lo}(d-1) - x_d +  \hat{\mu}_{in} - K)\end{split}
\end{equation*}
\sphinxAtStartPar
where \sphinxstyleemphasis{d} denotes a unit of time, \(x_d\) is the value of quantity
being monitored at time \(d\), \(\hat{\mu}_{in}\) is the
in\sphinxhyphen{}control mean of \(x_d\), and \(K\) is a “reference value”
related to the magnitude of change that one is interested in detecting.
\(S_{hi}\) and \(S_{lo}\) are the cumulative sum of positive and
negative changes. To detect a change in the observed values from the
in\sphinxhyphen{}control mean, the CUSUM scheme accumulates deviations that are
\(K\) units away from the in\sphinxhyphen{}control mean. Let \(\sigma_{in}\)
denote the in\sphinxhyphen{}control standard deviation of \(x_d\).

\sphinxstepscope


\section{CUSUM}
\label{\detokenize{ref_cusum:module-package.cusum}}\label{\detokenize{ref_cusum:cusum}}\label{\detokenize{ref_cusum::doc}}\index{module@\spxentry{module}!package.cusum@\spxentry{package.cusum}}\index{package.cusum@\spxentry{package.cusum}!module@\spxentry{module}}
\sphinxAtStartPar
Cumulative Sum (CUSUM)

\sphinxAtStartPar
@author: smriti.prathapan
\index{CUSUM (class in package.cusum)@\spxentry{CUSUM}\spxextra{class in package.cusum}}

\begin{fulllineitems}
\phantomsection\label{\detokenize{ref_cusum:package.cusum.CUSUM}}
\pysigstartsignatures
\pysigline
{\sphinxbfcode{\sphinxupquote{class\DUrole{w}{ }}}\sphinxcode{\sphinxupquote{package.cusum.}}\sphinxbfcode{\sphinxupquote{CUSUM}}}
\pysigstopsignatures
\sphinxAtStartPar
CUSUM class and its functionalities.
\index{change\_detection() (package.cusum.CUSUM method)@\spxentry{change\_detection()}\spxextra{package.cusum.CUSUM method}}

\begin{fulllineitems}
\phantomsection\label{\detokenize{ref_cusum:package.cusum.CUSUM.change_detection}}
\pysigstartsignatures
\pysiglinewithargsret
{\sphinxbfcode{\sphinxupquote{change\_detection}}}
{\sphinxparam{\DUrole{n}{normalized\_ref\_value}\DUrole{p}{:}\DUrole{w}{ }\DUrole{n}{float}\DUrole{w}{ }\DUrole{o}{=}\DUrole{w}{ }\DUrole{default_value}{0.5}}\sphinxparamcomma \sphinxparam{\DUrole{n}{normalized\_threshold}\DUrole{p}{:}\DUrole{w}{ }\DUrole{n}{float}\DUrole{w}{ }\DUrole{o}{=}\DUrole{w}{ }\DUrole{default_value}{4}}}
{{ $\rightarrow$ None}}
\pysigstopsignatures
\sphinxAtStartPar
Detects a change in the process.
\begin{quote}\begin{description}
\sphinxlineitem{Parameters}\begin{itemize}
\item {} 
\sphinxAtStartPar
\sphinxstyleliteralstrong{\sphinxupquote{pre\_change\_days}} (\sphinxstyleliteralemphasis{\sphinxupquote{int}}) \textendash{} Number of days for in\sphinxhyphen{}control phase.

\item {} 
\sphinxAtStartPar
\sphinxstyleliteralstrong{\sphinxupquote{normalized\_ref\_value}} (\sphinxstyleliteralemphasis{\sphinxupquote{float}}\sphinxstyleliteralemphasis{\sphinxupquote{, }}\sphinxstyleliteralemphasis{\sphinxupquote{optional}}) \textendash{} Normalized reference value for detecting a unit standard deviation change in mean of the process. Defaults to 0.5.

\item {} 
\sphinxAtStartPar
\sphinxstyleliteralstrong{\sphinxupquote{normalized\_threshold}} (\sphinxstyleliteralemphasis{\sphinxupquote{float}}\sphinxstyleliteralemphasis{\sphinxupquote{, }}\sphinxstyleliteralemphasis{\sphinxupquote{optional}}) \textendash{} Normalized threshold. Defaults to 4.

\end{itemize}

\end{description}\end{quote}

\end{fulllineitems}

\index{compute\_cusum() (package.cusum.CUSUM method)@\spxentry{compute\_cusum()}\spxextra{package.cusum.CUSUM method}}

\begin{fulllineitems}
\phantomsection\label{\detokenize{ref_cusum:package.cusum.CUSUM.compute_cusum}}
\pysigstartsignatures
\pysiglinewithargsret
{\sphinxbfcode{\sphinxupquote{compute\_cusum}}}
{\sphinxparam{\DUrole{n}{x}\DUrole{p}{:}\DUrole{w}{ }\DUrole{n}{list\DUrole{p}{{[}}float\DUrole{p}{{]}}}}\sphinxparamcomma \sphinxparam{\DUrole{n}{mu\_0}\DUrole{p}{:}\DUrole{w}{ }\DUrole{n}{float}}\sphinxparamcomma \sphinxparam{\DUrole{n}{k}\DUrole{p}{:}\DUrole{w}{ }\DUrole{n}{float}}}
{{ $\rightarrow$ tuple\DUrole{p}{{[}}list\DUrole{p}{{[}}float\DUrole{p}{{]}}\DUrole{p}{,}\DUrole{w}{ }list\DUrole{p}{{[}}float\DUrole{p}{{]}}\DUrole{p}{,}\DUrole{w}{ }list\DUrole{p}{{[}}float\DUrole{p}{{]}}\DUrole{p}{{]}}}}
\pysigstopsignatures
\sphinxAtStartPar
Compute CUSUM for the observations in x
\begin{quote}\begin{description}
\sphinxlineitem{Parameters}\begin{itemize}
\item {} 
\sphinxAtStartPar
\sphinxstyleliteralstrong{\sphinxupquote{x}} (\sphinxstyleliteralemphasis{\sphinxupquote{list}}\sphinxstyleliteralemphasis{\sphinxupquote{{[}}}\sphinxstyleliteralemphasis{\sphinxupquote{float}}\sphinxstyleliteralemphasis{\sphinxupquote{{]}}}) \textendash{} Performance metric to be monitored

\item {} 
\sphinxAtStartPar
\sphinxstyleliteralstrong{\sphinxupquote{mu\_0}} (\sphinxstyleliteralemphasis{\sphinxupquote{float}}) \textendash{} In\sphinxhyphen{}control mean of the observations/performance metric

\item {} 
\sphinxAtStartPar
\sphinxstyleliteralstrong{\sphinxupquote{k}} (\sphinxstyleliteralemphasis{\sphinxupquote{float}}) \textendash{} Reference value related to the magnitude of change that one is interested in detecting

\end{itemize}

\sphinxlineitem{Returns}
\sphinxAtStartPar
Positive cumulative sum, negative cumulative sum, and CUSUM

\sphinxlineitem{Return type}
\sphinxAtStartPar
tuple{[}list{[}float{]}, list{[}float{]}, list{[}float{]}{]}

\end{description}\end{quote}

\end{fulllineitems}

\index{initialize() (package.cusum.CUSUM method)@\spxentry{initialize()}\spxextra{package.cusum.CUSUM method}}

\begin{fulllineitems}
\phantomsection\label{\detokenize{ref_cusum:package.cusum.CUSUM.initialize}}
\pysigstartsignatures
\pysiglinewithargsret
{\sphinxbfcode{\sphinxupquote{initialize}}}
{}
{{ $\rightarrow$ None}}
\pysigstopsignatures
\sphinxAtStartPar
Initialize with the configuration file.

\end{fulllineitems}

\index{plot\_cusum\_plotly() (package.cusum.CUSUM method)@\spxentry{plot\_cusum\_plotly()}\spxextra{package.cusum.CUSUM method}}

\begin{fulllineitems}
\phantomsection\label{\detokenize{ref_cusum:package.cusum.CUSUM.plot_cusum_plotly}}
\pysigstartsignatures
\pysiglinewithargsret
{\sphinxbfcode{\sphinxupquote{plot\_cusum\_plotly}}}
{}
{{ $\rightarrow$ Figure}}
\pysigstopsignatures
\sphinxAtStartPar
Plot CUSUM value using Plotly
\begin{quote}\begin{description}
\sphinxlineitem{Returns}
\sphinxAtStartPar
CUSUM plot using Plotly graph object.

\sphinxlineitem{Return type}
\sphinxAtStartPar
go.Figure

\end{description}\end{quote}

\end{fulllineitems}

\index{plot\_input\_metric\_plotly() (package.cusum.CUSUM method)@\spxentry{plot\_input\_metric\_plotly()}\spxextra{package.cusum.CUSUM method}}

\begin{fulllineitems}
\phantomsection\label{\detokenize{ref_cusum:package.cusum.CUSUM.plot_input_metric_plotly}}
\pysigstartsignatures
\pysiglinewithargsret
{\sphinxbfcode{\sphinxupquote{plot\_input\_metric\_plotly}}}
{}
{{ $\rightarrow$ Figure}}
\pysigstopsignatures
\sphinxAtStartPar
Plot the input metric using Plotly.
\begin{quote}\begin{description}
\sphinxlineitem{Returns}
\sphinxAtStartPar
Scatter plot as Plotly graph object.

\sphinxlineitem{Return type}
\sphinxAtStartPar
go.Figure

\end{description}\end{quote}

\end{fulllineitems}

\index{set\_df\_metric\_csv() (package.cusum.CUSUM method)@\spxentry{set\_df\_metric\_csv()}\spxextra{package.cusum.CUSUM method}}

\begin{fulllineitems}
\phantomsection\label{\detokenize{ref_cusum:package.cusum.CUSUM.set_df_metric_csv}}
\pysigstartsignatures
\pysiglinewithargsret
{\sphinxbfcode{\sphinxupquote{set\_df\_metric\_csv}}}
{\sphinxparam{\DUrole{n}{data\_csv}\DUrole{p}{:}\DUrole{w}{ }\DUrole{n}{DataFrame}}}
{{ $\rightarrow$ None}}
\pysigstopsignatures
\sphinxAtStartPar
Assign the performance metric data to be used for CUSUM.
\begin{quote}\begin{description}
\sphinxlineitem{Parameters}
\sphinxAtStartPar
\sphinxstyleliteralstrong{\sphinxupquote{data\_csv}} (\sphinxstyleliteralemphasis{\sphinxupquote{DataFrame}}\sphinxstyleliteralemphasis{\sphinxupquote{ or }}\sphinxstyleliteralemphasis{\sphinxupquote{TextFileReader}}) \textendash{} A comma\sphinxhyphen{}separated values (csv) file is returned as two\sphinxhyphen{}dimensional data structure with labeled axes.

\end{description}\end{quote}

\end{fulllineitems}

\index{set\_df\_metric\_default() (package.cusum.CUSUM method)@\spxentry{set\_df\_metric\_default()}\spxextra{package.cusum.CUSUM method}}

\begin{fulllineitems}
\phantomsection\label{\detokenize{ref_cusum:package.cusum.CUSUM.set_df_metric_default}}
\pysigstartsignatures
\pysiglinewithargsret
{\sphinxbfcode{\sphinxupquote{set\_df\_metric\_default}}}
{}
{{ $\rightarrow$ None}}
\pysigstopsignatures
\sphinxAtStartPar
Read the provided performance metric data to be used for CUSUM for an example.

\end{fulllineitems}

\index{set\_init\_stats() (package.cusum.CUSUM method)@\spxentry{set\_init\_stats()}\spxextra{package.cusum.CUSUM method}}

\begin{fulllineitems}
\phantomsection\label{\detokenize{ref_cusum:package.cusum.CUSUM.set_init_stats}}
\pysigstartsignatures
\pysiglinewithargsret
{\sphinxbfcode{\sphinxupquote{set\_init\_stats}}}
{\sphinxparam{\DUrole{n}{init\_days}\DUrole{p}{:}\DUrole{w}{ }\DUrole{n}{int}}}
{{ $\rightarrow$ None}}
\pysigstopsignatures
\sphinxAtStartPar
Use initial days to calculate in\sphinxhyphen{}control mean and standard deviation.
\begin{quote}\begin{description}
\sphinxlineitem{Parameters}
\sphinxAtStartPar
\sphinxstyleliteralstrong{\sphinxupquote{init\_days}} (\sphinxstyleliteralemphasis{\sphinxupquote{int}}\sphinxstyleliteralemphasis{\sphinxupquote{, }}\sphinxstyleliteralemphasis{\sphinxupquote{optional}}) \textendash{} Initial days when observations are considered stable. Defaults to 30.

\end{description}\end{quote}

\end{fulllineitems}

\index{set\_timeline() (package.cusum.CUSUM method)@\spxentry{set\_timeline()}\spxextra{package.cusum.CUSUM method}}

\begin{fulllineitems}
\phantomsection\label{\detokenize{ref_cusum:package.cusum.CUSUM.set_timeline}}
\pysigstartsignatures
\pysiglinewithargsret
{\sphinxbfcode{\sphinxupquote{set\_timeline}}}
{\sphinxparam{\DUrole{n}{data}\DUrole{p}{:}\DUrole{w}{ }\DUrole{n}{ndarray}}}
{{ $\rightarrow$ None}}
\pysigstopsignatures
\sphinxAtStartPar
Set the timeline of observations.
\begin{quote}\begin{description}
\sphinxlineitem{Parameters}
\sphinxAtStartPar
\sphinxstyleliteralstrong{\sphinxupquote{data}} (\sphinxstyleliteralemphasis{\sphinxupquote{np.ndarray}}) \textendash{} Data of the metric values across the observations.

\end{description}\end{quote}

\end{fulllineitems}


\end{fulllineitems}


\sphinxstepscope


\section{ARLTheoretical}
\label{\detokenize{ref_theoretical:module-package.ARLTheoretical}}\label{\detokenize{ref_theoretical:arltheoretical}}\label{\detokenize{ref_theoretical::doc}}\index{module@\spxentry{module}!package.ARLTheoretical@\spxentry{package.ARLTheoretical}}\index{package.ARLTheoretical@\spxentry{package.ARLTheoretical}!module@\spxentry{module}}
\sphinxAtStartPar
ARLTheoretical

\sphinxAtStartPar
@author: smriti.prathapan
\index{get\_ARL\_1() (in module package.ARLTheoretical)@\spxentry{get\_ARL\_1()}\spxextra{in module package.ARLTheoretical}}

\begin{fulllineitems}
\phantomsection\label{\detokenize{ref_theoretical:package.ARLTheoretical.get_ARL_1}}
\pysigstartsignatures
\pysiglinewithargsret
{\sphinxcode{\sphinxupquote{package.ARLTheoretical.}}\sphinxbfcode{\sphinxupquote{get\_ARL\_1}}}
{\sphinxparam{\DUrole{n}{h}\DUrole{p}{:}\DUrole{w}{ }\DUrole{n}{float}}\sphinxparamcomma \sphinxparam{\DUrole{n}{shift\_in\_mean}\DUrole{p}{:}\DUrole{w}{ }\DUrole{n}{list\DUrole{p}{{[}}float\DUrole{p}{{]}}}}\sphinxparamcomma \sphinxparam{\DUrole{n}{dict\_ARL0\_k}\DUrole{p}{:}\DUrole{w}{ }\DUrole{n}{OrderedDict}}}
{{ $\rightarrow$ DataFrame}}
\pysigstopsignatures
\sphinxAtStartPar
Get the ARL1 along with k values.
\begin{quote}\begin{description}
\sphinxlineitem{Parameters}\begin{itemize}
\item {} 
\sphinxAtStartPar
\sphinxstyleliteralstrong{\sphinxupquote{h}} (\sphinxstyleliteralemphasis{\sphinxupquote{float}}) \textendash{} Normalized threshold.

\item {} 
\sphinxAtStartPar
\sphinxstyleliteralstrong{\sphinxupquote{shift\_in\_mean}} (\sphinxstyleliteralemphasis{\sphinxupquote{list}}\sphinxstyleliteralemphasis{\sphinxupquote{{[}}}\sphinxstyleliteralemphasis{\sphinxupquote{float}}\sphinxstyleliteralemphasis{\sphinxupquote{{]}}}) \textendash{} List of the values of shift in mean.

\item {} 
\sphinxAtStartPar
\sphinxstyleliteralstrong{\sphinxupquote{dict\_ARL0\_k}} (\sphinxstyleliteralemphasis{\sphinxupquote{OrderedDict}}) \textendash{} Data dictionary of ARL0 and k

\end{itemize}

\sphinxlineitem{Returns}
\sphinxAtStartPar
Table for ARL1 and k values.

\sphinxlineitem{Return type}
\sphinxAtStartPar
pd.DataFrame

\end{description}\end{quote}

\end{fulllineitems}

\index{get\_ARL\_1\_h\_mu1\_k() (in module package.ARLTheoretical)@\spxentry{get\_ARL\_1\_h\_mu1\_k()}\spxextra{in module package.ARLTheoretical}}

\begin{fulllineitems}
\phantomsection\label{\detokenize{ref_theoretical:package.ARLTheoretical.get_ARL_1_h_mu1_k}}
\pysigstartsignatures
\pysiglinewithargsret
{\sphinxcode{\sphinxupquote{package.ARLTheoretical.}}\sphinxbfcode{\sphinxupquote{get\_ARL\_1\_h\_mu1\_k}}}
{\sphinxparam{\DUrole{n}{h}\DUrole{p}{:}\DUrole{w}{ }\DUrole{n}{float}}\sphinxparamcomma \sphinxparam{\DUrole{n}{k}\DUrole{p}{:}\DUrole{w}{ }\DUrole{n}{float}}\sphinxparamcomma \sphinxparam{\DUrole{n}{mu1}\DUrole{p}{:}\DUrole{w}{ }\DUrole{n}{float}}}
{{ $\rightarrow$ float}}
\pysigstopsignatures
\sphinxAtStartPar
Calculate ARL\_1 with given Shift in Mean (mu1) and k.
\begin{quote}\begin{description}
\sphinxlineitem{Parameters}\begin{itemize}
\item {} 
\sphinxAtStartPar
\sphinxstyleliteralstrong{\sphinxupquote{h}} (\sphinxstyleliteralemphasis{\sphinxupquote{float}}) \textendash{} Normalized threshold.

\item {} 
\sphinxAtStartPar
\sphinxstyleliteralstrong{\sphinxupquote{k}} (\sphinxstyleliteralemphasis{\sphinxupquote{float}}) \textendash{} Normalized reference value.

\item {} 
\sphinxAtStartPar
\sphinxstyleliteralstrong{\sphinxupquote{mu1}} (\sphinxstyleliteralemphasis{\sphinxupquote{float}}) \textendash{} Intended shift in mean.

\end{itemize}

\sphinxlineitem{Returns}
\sphinxAtStartPar
Detection delay (ARL1).

\sphinxlineitem{Return type}
\sphinxAtStartPar
float

\end{description}\end{quote}

\end{fulllineitems}

\index{get\_ref\_value() (in module package.ARLTheoretical)@\spxentry{get\_ref\_value()}\spxextra{in module package.ARLTheoretical}}

\begin{fulllineitems}
\phantomsection\label{\detokenize{ref_theoretical:package.ARLTheoretical.get_ref_value}}
\pysigstartsignatures
\pysiglinewithargsret
{\sphinxcode{\sphinxupquote{package.ARLTheoretical.}}\sphinxbfcode{\sphinxupquote{get\_ref\_value}}}
{\sphinxparam{\DUrole{n}{h}\DUrole{p}{:}\DUrole{w}{ }\DUrole{n}{float}}\sphinxparamcomma \sphinxparam{\DUrole{n}{list\_ARL\_0}\DUrole{p}{:}\DUrole{w}{ }\DUrole{n}{list\DUrole{p}{{[}}float\DUrole{p}{{]}}}}}
{{ $\rightarrow$ tuple\DUrole{p}{{[}}DataFrame\DUrole{p}{,}\DUrole{w}{ }OrderedDict\DUrole{p}{{]}}}}
\pysigstopsignatures
\sphinxAtStartPar
provides normalized reference values k for provided list of ARL0, given the value of normalized threshold h.
\begin{quote}\begin{description}
\sphinxlineitem{Parameters}\begin{itemize}
\item {} 
\sphinxAtStartPar
\sphinxstyleliteralstrong{\sphinxupquote{h}} (\sphinxstyleliteralemphasis{\sphinxupquote{float}}) \textendash{} Normalized threshold.

\item {} 
\sphinxAtStartPar
\sphinxstyleliteralstrong{\sphinxupquote{list\_ARL\_0}} (\sphinxstyleliteralemphasis{\sphinxupquote{list}}) \textendash{} List of ARL0 values.

\end{itemize}

\sphinxlineitem{Returns}
\sphinxAtStartPar
Dataframe of ARL0 and k, Data dictionary of ARL0 and k; where k is normalized reference value.

\sphinxlineitem{Return type}
\sphinxAtStartPar
tuple{[}pd.Dataframe, OrderedDict{]}

\end{description}\end{quote}

\end{fulllineitems}

\index{get\_ref\_value\_k() (in module package.ARLTheoretical)@\spxentry{get\_ref\_value\_k()}\spxextra{in module package.ARLTheoretical}}

\begin{fulllineitems}
\phantomsection\label{\detokenize{ref_theoretical:package.ARLTheoretical.get_ref_value_k}}
\pysigstartsignatures
\pysiglinewithargsret
{\sphinxcode{\sphinxupquote{package.ARLTheoretical.}}\sphinxbfcode{\sphinxupquote{get\_ref\_value\_k}}}
{\sphinxparam{\DUrole{n}{h}\DUrole{p}{:}\DUrole{w}{ }\DUrole{n}{float}}\sphinxparamcomma \sphinxparam{\DUrole{n}{ARL\_0}\DUrole{p}{:}\DUrole{w}{ }\DUrole{n}{float}}}
{{ $\rightarrow$ float}}
\pysigstopsignatures
\sphinxAtStartPar
Calculation for the reference value for given h and ARL\_0.
\begin{quote}\begin{description}
\sphinxlineitem{Parameters}\begin{itemize}
\item {} 
\sphinxAtStartPar
\sphinxstyleliteralstrong{\sphinxupquote{h}} (\sphinxstyleliteralemphasis{\sphinxupquote{float}}) \textendash{} Normalized threshold.

\item {} 
\sphinxAtStartPar
\sphinxstyleliteralstrong{\sphinxupquote{ARL\_0}} (\sphinxstyleliteralemphasis{\sphinxupquote{float}}) \textendash{} ARL0 value.

\end{itemize}

\sphinxlineitem{Returns}
\sphinxAtStartPar
Normalized reference value k.

\sphinxlineitem{Return type}
\sphinxAtStartPar
float

\end{description}\end{quote}

\end{fulllineitems}


\sphinxstepscope


\section{Utils}
\label{\detokenize{ref_utils:module-package.utils}}\label{\detokenize{ref_utils:utils}}\label{\detokenize{ref_utils::doc}}\index{module@\spxentry{module}!package.utils@\spxentry{package.utils}}\index{package.utils@\spxentry{package.utils}!module@\spxentry{module}}
\sphinxAtStartPar
Utilities to handle different operations
\index{get\_greattable\_as\_html() (in module package.utils)@\spxentry{get\_greattable\_as\_html()}\spxextra{in module package.utils}}

\begin{fulllineitems}
\phantomsection\label{\detokenize{ref_utils:package.utils.get_greattable_as_html}}
\pysigstartsignatures
\pysiglinewithargsret
{\sphinxcode{\sphinxupquote{package.utils.}}\sphinxbfcode{\sphinxupquote{get\_greattable\_as\_html}}}
{\sphinxparam{\DUrole{n}{df}\DUrole{p}{:}\DUrole{w}{ }\DUrole{n}{DataFrame}}}
{{ $\rightarrow$ GT}}
\pysigstopsignatures
\sphinxAtStartPar
Get the great\_table as HTML from Pandas dataframe.
\begin{quote}\begin{description}
\sphinxlineitem{Parameters}
\sphinxAtStartPar
\sphinxstyleliteralstrong{\sphinxupquote{df}} (\sphinxstyleliteralemphasis{\sphinxupquote{pd.DataFrame}}) \textendash{} Dataframe to rendera as a table.

\sphinxlineitem{Returns}
\sphinxAtStartPar
Table in HTML format.

\sphinxlineitem{Return type}
\sphinxAtStartPar
gt.GT

\end{description}\end{quote}

\end{fulllineitems}

\index{populate\_summary\_table\_ARL0\_k() (in module package.utils)@\spxentry{populate\_summary\_table\_ARL0\_k()}\spxextra{in module package.utils}}

\begin{fulllineitems}
\phantomsection\label{\detokenize{ref_utils:package.utils.populate_summary_table_ARL0_k}}
\pysigstartsignatures
\pysiglinewithargsret
{\sphinxcode{\sphinxupquote{package.utils.}}\sphinxbfcode{\sphinxupquote{populate\_summary\_table\_ARL0\_k}}}
{\sphinxparam{\DUrole{n}{summary\_table\_df\_ARL0\_k}\DUrole{p}{:}\DUrole{w}{ }\DUrole{n}{DataFrame}}\sphinxparamcomma \sphinxparam{\DUrole{n}{h}}}
{{ $\rightarrow$ GT}}
\pysigstopsignatures
\sphinxAtStartPar
Populate ARLTheoretical.summary\_table\_df\_ARL0\_k.
\begin{quote}\begin{description}
\sphinxlineitem{Parameters}\begin{itemize}
\item {} 
\sphinxAtStartPar
\sphinxstyleliteralstrong{\sphinxupquote{summary\_table\_df\_ARL0\_k}} (\sphinxstyleliteralemphasis{\sphinxupquote{pd.DataFrame}}) \textendash{} Dataframe of ARL0 and its respective values of k.

\item {} 
\sphinxAtStartPar
\sphinxstyleliteralstrong{\sphinxupquote{h}} (\sphinxstyleliteralemphasis{\sphinxupquote{float}}) \textendash{} Normalized threshold.

\end{itemize}

\sphinxlineitem{Returns}
\sphinxAtStartPar
Table of ARL0 and k in HTML format.

\sphinxlineitem{Return type}
\sphinxAtStartPar
gt.GT

\end{description}\end{quote}

\end{fulllineitems}

\index{populate\_summary\_table\_ARL1\_k() (in module package.utils)@\spxentry{populate\_summary\_table\_ARL1\_k()}\spxextra{in module package.utils}}

\begin{fulllineitems}
\phantomsection\label{\detokenize{ref_utils:package.utils.populate_summary_table_ARL1_k}}
\pysigstartsignatures
\pysiglinewithargsret
{\sphinxcode{\sphinxupquote{package.utils.}}\sphinxbfcode{\sphinxupquote{populate\_summary\_table\_ARL1\_k}}}
{\sphinxparam{\DUrole{n}{summary\_table\_df\_ARL1\_k}\DUrole{p}{:}\DUrole{w}{ }\DUrole{n}{DataFrame}}\sphinxparamcomma \sphinxparam{\DUrole{n}{dict\_ARL0\_k}\DUrole{p}{:}\DUrole{w}{ }\DUrole{n}{OrderedDict}}\sphinxparamcomma \sphinxparam{\DUrole{n}{h}}}
{{ $\rightarrow$ GT}}
\pysigstopsignatures
\sphinxAtStartPar
Populate Multiindex table specific for ARLTheoretical.summary\_table\_df\_ARL1\_k
\begin{quote}\begin{description}
\sphinxlineitem{Parameters}\begin{itemize}
\item {} 
\sphinxAtStartPar
\sphinxstyleliteralstrong{\sphinxupquote{summary\_table\_df\_ARL1\_k}} (\sphinxstyleliteralemphasis{\sphinxupquote{pd.DataFrame}}) \textendash{} Dataframe with ARL1 and k values.

\item {} 
\sphinxAtStartPar
\sphinxstyleliteralstrong{\sphinxupquote{dict\_ARL0\_k}} (\sphinxstyleliteralemphasis{\sphinxupquote{OrderedDict}}) \textendash{} Data Dictionary with the mapping between ARL0 and k.

\item {} 
\sphinxAtStartPar
\sphinxstyleliteralstrong{\sphinxupquote{h}} (\sphinxstyleliteralemphasis{\sphinxupquote{float}}) \textendash{} Normalized threshold.

\end{itemize}

\sphinxlineitem{Returns}
\sphinxAtStartPar
Table for ARL1 and k in HTML format.

\sphinxlineitem{Return type}
\sphinxAtStartPar
gt.GT

\end{description}\end{quote}

\end{fulllineitems}



\renewcommand{\indexname}{Python Module Index}
\begin{sphinxtheindex}
\let\bigletter\sphinxstyleindexlettergroup
\bigletter{p}
\item\relax\sphinxstyleindexentry{package.ARLTheoretical}\sphinxstyleindexpageref{ref_theoretical:\detokenize{module-package.ARLTheoretical}}
\item\relax\sphinxstyleindexentry{package.cusum}\sphinxstyleindexpageref{ref_cusum:\detokenize{module-package.cusum}}
\item\relax\sphinxstyleindexentry{package.utils}\sphinxstyleindexpageref{ref_utils:\detokenize{module-package.utils}}
\end{sphinxtheindex}

\renewcommand{\indexname}{Index}
\printindex
\end{document}